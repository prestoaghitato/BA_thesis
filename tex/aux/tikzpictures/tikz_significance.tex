% !TEX root = ../../scrreprt_figures/ba_scrreprt_figures.tex
% @author Marcel Ruland (2018) 247

% semantic commands
\newcommand{\nodefont}[1]{{\scriptsize #1}}
\newcommand{\commentfont}[1]{{\rmfamily\textit{#1}}}
\newcommand{\nullnode}{\nodefont{100 \textsc{ns}}}
\newcommand{\samplenode}{\nodefont{sample}}
\newcommand{\sampledummy}{{\color{white}\samplenode} \\[0.2em] {\color{white}\nodefont{10}}}
\newcommand{\nodecdots}{\nodefont{\(\cdots\)}}
\newcommand{\annotationcounter}[1]{{\rmfamily\footnotesize\addfontfeature{Numbers=Lining}\textbf{#1.~}}}

% coordinates
\newcommand{\nully}{8}			% y-position of null nodes
\newcommand{\sampley}{5}		% y-position of sample nodes
\newcommand{\realx}{6.5}			% x position of leftmost real node
\newcommand{\realy}{\nully}	% y-position of real nodes

% incrementers
\newcommand{\nullinc}{1.5}		% increment for null nodes
\newcommand{\sampleinc}{2}		% increment for sample nodes
\newcommand{\realinc}{1}		% increment for real nodes
\newcommand{\realtonullinc}{0.3}% increment for real to null arrows

% other
\newcommand{\nodeoffset}{0.5}		% offset for \cdots
\newcommand{\annotationwidth}{1.5cm}% width of annotations on left side
\pgfmathdeclarefunction{gauss}{2}{\pgfmathparse{1/(#2*sqrt(2*pi))*exp(-((x-#1)^2)/(2*#2^2))}}	% gaussian function for plot


\begin{tikzpicture}[
	every node/.append style={
		font=\sffamily\tiny\addfontfeature{
			Numbers=Lining,
			Letters=Uppercase,
		}
	},
	node distance = 2 and 0.2,
	align=center,
]
	
	
%%% NODES
	%% null nodes
	\node[draw] 	(null1)	{\nullnode \\[0.2em] \nodefont{1}};
	\node[draw, right=of null1]	(null2)	{\nullnode \\[0.2em] \nodefont{2}};
	\node[draw, right=of null2]	(null3)	{\nullnode \\[0.2em] \nodefont{3}};
	\node[draw, right=0.8 of null3]	(null10){\nullnode \\[0.2em] \nodefont{10}};
	
	%% real nodes
	\node[draw, right=0.6 of null10, circle, minimum size=2.5em]	(real1) {\nodefont{R1}};
	\node[draw, right=of real1, circle, minimum size=2.5em]	(real2) {\nodefont{R2}};
	\node[draw, right=of real2, circle, minimum size=2.5em]	(real3) {\nodefont{R3}};
	\node[draw, right=0.8 of real3, circle, minimum size=2.5em]	(real10) {\nodefont{R10}};
	% blind nodes for real sample node
	\node[below=of real1.center] (realsampleleft) {\sampledummy};
	\node[below=of real10.center] (realsampleright) {\sampledummy};
	% then draw it in the middle with a path
	\path(realsampleleft) --node[draw, circle, minimum size=2.5em] (realsample) {\nodefont{R1--10}} (realsampleright);  % real sample node
	
	%% sample nodes
	\node[draw, below=of null1.center] (sample1) {\samplenode \\[0.2em] \nodefont{1}};
	\node[draw, below=of null10.center] (sample100) {\samplenode \\[0.2em] \nodefont{100}};
	\path(sample1) --node[draw] (sample50) {\samplenode \\[0.2em] \nodefont{50}} (sample100);

	
	%% \cdots
	\path (null3) --node{\nodecdots} (null10);
	\path (sample1) --node{\nodecdots} (sample50);
	\path (sample50) --node{\nodecdots} (sample100);
	\path (real3) --node{\nodecdots} (real10);
	
	
%%% PLOT AND X
	% blind node for plot anchoring
	\node[below=3 of sample50.center] (plot) {};
	% le big red X
	\node[below=3 of realsample.center] (x) {{\huge\(\textcolor{graphred}{X}\)}};
	
	\begin{axis}[
		at={(plot)},  % set origin coordinate in tikzpicture
		anchor=center,
		scale=0.5,  % half as large
		xmin=-3,
		xmax=3,
		ymin=0,  % set ymin to 0
		xticklabels={,,},  % surpress digits at x-axis
		hide y axis,  % hide y axes (duh)
		axis x line*=bottom, % no box around the plot, only x and y axis
		every axis plot post/.append style={% all plots: from -2:2, 50 samples, smooth, no marks
			mark=none,
			domain=-3:3,
			samples=50,
			smooth
		}
	]
		% add normal curve
		\addplot[color=graphgreen] {gauss(0,0.75)};
		% add vertical line
		\draw[color=graphgreen, dashed] (axis cs:0,\pgfkeysvalueof{/pgfplots/ymin}) -- (axis cs:0,\pgfkeysvalueof{/pgfplots/ymax});
	\end{axis}
	
	% the big question
	\path(plot) --node (bigquestion) {\large\(\textcolor{graphgreen}{\mu} \stackrel{?}{=} \textcolor{graphred}{X}\)} (x);

	
%%% ARROWS
	%% to sample arrows
	\foreach \i in {1, 2, 3, 10}{
		% null
		\draw[-stealth, graphgreen] (null\i.south west)  to (sample1.north);
		\draw[-stealth, graphgreen] (null\i.south)  to (sample50.north);
		\draw[-stealth, graphgreen] (null\i.south east)  to (sample100.north);
		% real
		\draw[-stealth, graphred] (real\i.south)  to (realsample);
	}
		
	
	%% real to null arrows
	\foreach \i/\j in {0.2/1, 0.4/2, 0.6/3, 0.8/10}{
		% blind nodes above real nodes
		\node[above=\i of real\j] (realb\j) {};
		% actual arrows
		\draw[-stealth, graphred] (real\j) -- (realb\j.center) -| (null\j);
	}
	% extra blind note for annotation
	\node[above=0.82 of null10] (nullb10) {};
	
	%% sample to plot/X arrows
	\foreach \i in {1, 50, 100}
		\draw[-stealth, graphgreen] (sample\i.south)  to ([shift={(0,1.5)}]plot);
	\draw[-stealth, graphred] (realsample) to (x.north);
	

	
%%% ANNOTATIONS
	% real to null
	\path (nullb10) --node[above]{\annotationcounter{1}\commentfont{generate 100 null sequences from each real sequence}} (realb10);

	% null to sample
	\draw[decorate,	decoration={brace,amplitude=5pt}]
	(sample1.north west) --
	node[left, xshift=-0.5em, align=right, text width=\annotationwidth]
	{\annotationcounter{2}\commentfont{\emph{i}-th null sequence goes into \emph{i}-th sample}}
	(null1.south west);
	
	% apply fpm
	\draw[decorate,	decoration={brace,amplitude=5pt}]
	(sample1.south west) --
	node[left, xshift=-0.5em, align=right, text width=\annotationwidth]
	{\annotationcounter{3}\commentfont{apply \textsc{fpm} to all samples separately and to all real sequences together}}
	(sample1.north west);
	(null1.south west);
	
	% distribution
	\draw[decorate,	decoration={brace,amplitude=5pt}]
	([shift={(0,-4.2)}]sample1.south west) --
	node[left, xshift=-0.5em, align=right, text width=\annotationwidth]
	{\annotationcounter{4}\commentfont{for metric~\emph{m} of rule~\emph{r}, a null distribution of \(\leq\)~100 values is formed}}
	(sample1.south west);

	% the big question
	\node[above=0 of bigquestion, align=left, text width=2.5cm]
	{\annotationcounter{5}\commentfont{test the real observation of metric~\emph{m} in rule~\emph{r} for significance against the corresponding null distribution}};
\end{tikzpicture}