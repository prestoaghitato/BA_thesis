% !TEX root = ../scrreprt/ba_scrreprt_master.tex
% @author Marcel Ruland (2018)

\chapter{Conclusion}
\label{ch:con}
% theoretical aim
Aside from the methodological aim of this thesis, another more theoretical aim has been to provide evidence in favour of the fact that turn-taking is a multimodal phenomenon that cannot be meaningfully observed by only considering the linguistic modality.
Even though a variety of literature approaching turn-taking as a multimodal phenomenon exists (cf.~subsection~\ref{ssec:introrestt}), the vast majority still assumes unimodality.
Section~\ref{sig:res} has indeed provided such empirical evidence, showing a variety of multimodal rules with high and significant confidence, \noc, or duration.
Most of the rules with the highest significant values involve gazing and/or smiling at the interaction partner, further underlining the central role of gaze in early mother-infant interaction and perhaps indicating that smile also plays an important role as it accompanied mothers' speech and gazing at their infant, presumably to capture or maintain their infant's attention \citep{nomikou_educating_2013,nomikou_constructing_2016}.

% methodological aim
Methodologically, the first step taken was to consider reaction times when mining rules.
This was done by demanding a minimum delay of 289~\ms\ or 850~\ms\ when the mother and the infant respectively were reacting to a behaviour.
As was stated in the paragraphs on reaction time in interaction in subsection~\ref{ssec:fpmmetapp}, the length of these delays needs further empirical verification.
As of now, the mining approach did not produce any counterintuitive rules, but the main issue shown when modelling the \rt s is that there is a gap in the literature concerning \rt\ in interaction.
The main methodological aim of this thesis (as outlined in section~\ref{sec:introaim}) was to improve the objectiveness and quantitativeness of choosing relevant or \emph{interesting} rules when using \fpmlower\ to mine temporal relationships in human behaviour in interaction.
This was achieved by introducing quantitative support in the form of statistical significance to all rules.

The results given here are a first exploratory summary.
But given the fact that a total of 358~rules have shown significant values in at least one metric (counting identical rules with different delays twice), the data is in all probability far from exhausted.
It can be further probed in several different ways.
First off, the data may assist in a more traditional way by disproving or affirming concrete hypotheses.
This has already been done here in the form of confirming existing research by \citet{nomikou_educating_2013}.
Another way to make further use of the data is to generate new hypotheses from the rules which can then be tested empirically.
This has already been alluded to in the first paragraph of this chapter.\showcomment{Still first paragraph?}

%formulate some hypotheses from the results
\paragraph{Future Research}
%\label{sec:confut}
%Fourrier transforms
%road to turn-taking
%no child, 2 equal partners
%way more sophisticated models, taking into account every ant --> suc pair, potentially also combinations
%
%we need to look at \rt in conversation/interaction
%
%longitudinal studies, \textsc{tt} development, turns slow down around 12months \citep{hilbrink_early_2015}
%significantly \emph{lower} stuff
%incorporate gesture
%conditional probability needed for evaluation