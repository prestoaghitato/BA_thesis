% !TEX root = ../scrreprt/ba_scrreprt_master.tex
% @author Marcel Ruland (2018)

\chapter{\introduction}
\label{ch:intro}
When we speak, we do not all speak at once.
Humans organise their conversations in turns and tend to speak one after another much rather than all at the same time.
A \emph{turn} is a basic unit of speech in conversation \citep{sacks_simplest_1978}.
When a speaker begins their turn, they gain the right to speak and\dash upon finishing their utterance\dash pass that right on to another participant in the conversation.
A turn is on average 2~\s\ in length, with high variation.
It usually consists of syntactically complete clauses and gives pragmatically sufficient information, though this is not a necessity \citep[\pnum{8}]{levinson_turn-taking_2016}.
\emph{Turn-taking} is the act of one speaker finishing their turn and another speaker beginning their turn, i.e.\ passing on the right to speak from one speaker to another.
Turns and turn-taking must not necessarily be strictly linguistic (in the sense of making use of language only), but may make use of other modalities, for example gestures \citep[\pnum{28--43}]{schmitt_zur_2005}.
In fact, children regularly engage in turn-taking before having acquired their first word \citep[\pnum{1311}]{casillas_turn-taking_2016}.
This means that turn-taking is not dependent on language and is therefore not intrinsically linguistic in and of itself.

If turn-taking is not dependant on the linguistic modality but may make use of it, then it logically follows that turn-taking is a multimodal phenomenon.\footnote{%
Assuming there is no such thing as non-modal turn-taking.}
Despite this fact, turn-taking has so far mostly been considered as unimodal in the literature \citep[\pnum{1}]{rohlfing_multimodal_underreview}.
The downside of investigating only one modality of a multimodal phenomenon is\dash quite obviously\dash that it ``omits other behaviors relevant to the exchange'' \citep[\pnum{3}]{rohlfing_multimodal_underreview}.

\citet{rohlfing_multimodal_underreview} have identified turn-taking as a multimodal phenomenon; taking into consideration linguistic utterances, non-linguistic vocalisations, gaze, and smile.
They approached the phenomenon by applying \fpmlower, a technique at the intersection of statistics and computer science, to a corpus of 10 mother-infant dyads.
The fundamental aim of \fpmlower\ is to find patterns in a data set that occur \emph{frequently}.
The specific kind of pattern and the definition of \emph{frequent} depend heavily on the individual application and cannot be meaningfully generalised.
In contrast to descriptive and inferential statistics\dash which disprove or affirm hypotheses generated beforehand\dash \fpmlower\ is of an exploratory, hypothesis-generating nature.
Depending on the patterns found, one may then formulate hypotheses based on these found patterns, which may then in turn be tested empirically using descriptive and inferential statistics \cite[\pnum{6~ff.,~tba}]{rohlfing_multimodal_underreview,han_data_2012}.\showcomment{ISSUE: How do you give page references for each individual source when citing multiple sources?}
The basic concepts of \fpmlower\ are explained by \citet[\pnum{57}, emphasis in original]{han_frequent_2007}:
\begin{customquote}
``Frequent patterns are itemsets, subsequences, or substructures that appear in a data set with frequency no less than a user-specified threshold.
For example, a set of items, such as milk and bread, that appear frequently together in a transaction data set, is a \emph{frequent itemset}.
A subsequence, such as buying first a PC, then a digital camera, and then a memory card, if it occurs frequently in a shopping history database, is a \emph{(frequent) sequential pattern.}''
\end{customquote}

\paragraph{Organisation of this thesis}
The present thesis aims to strengthen the statistical objectiveness of applying \fpmlower\ to human interaction by introducing a notion of significance.
It is organised into four main chapters.
Chapter~\ref{ch:intro} (i.e.\ the current chapter) will proceed by motivating the need to investigate turn-taking\dash especially from an evolutionary point of view\dash in section~\ref{sec:intrott} and then give an overview of relevant previous research in section \ref{sec:introres}.
The last section of the first chapter (\ref{sec:introaim}) further illustrates the aim of the present thesis, which has already been alluded to at the beginning of this paragraph.\showcomment{Still this paragraph?}

The second chapter is almost entirely of an methodological nature.
Section~\ref{sec:fpmmet} explains the methodology of \citet{rohlfing_multimodal_underreview} in detail, which is then further adapted to the specific interactional nature of the data by modelling reaction times in subsection~\ref{ssec:fpmmetapp}.
Section~\ref{sec:fpmmetr} introduces the three metrics\dash confidence, duration, and number of occurrences\dash which are used for evaluating rules\footnote{The notion of a \emph{rule} will be explained in subsection \ref{ssec:fpmmetapp}.}.
Chapter~\ref{ch:fpm}'s last section briefly contrasts the results of \citet{rohlfing_multimodal_underreview} with the new results taking reaction time into account.
This section is not an extensive discussion, which is reserved for the end of chapter~\ref{ch:sig}.

Chapter~\ref{ch:sig} then forms the main part of the thesis\dash not in length but in contribution.
In this chapter, a notion of statistical significance is introduced to the rules mined in the second chapter.
It is divided into four sections.
Section~\ref{sec:signat} conceptually explains the meaning of statistical significance and what it can and cannot do.
Section~\ref{sec:sigmet} explains how exactly significance was calculated by creating null sequences for comparison against the real data; followed by section~\ref{sec:sigmetr}, which briefly goes over the three metrics again, this time in the context of significance and its semantic interpretation.
The chapter closes with an extensive results section.
The thesis itself closes with chapter~\ref{ch:con}, the conclusion.
Finally, an appendix lists all rules mined and their corresponding values of all metrics.


\section{Why is Turn-Taking important?}
\label{sec:intrott}
According to present knowledge, turn-taking is a universal feature of human language, a \emph{language universal.}
A human language that does not make use of turn-taking has not been found to date and it is unlikely that such a language will ever be found (although\dash as is always the case in empirical science\dash this is not impossible).
Its implications to language processing and production are fundamental as it dictates the temporal organisation of the information flow between humans.
Turn-taking may be characterised as a temporal phenomenon.
Its temporal structure shows high variation in length of individual turns, which average at 2~seconds \citep[\pnum{7}]{levinson_turn-taking_2016}, but little variation in the gaps between turns, whose modal length in American English telephone conversations is 200~\ms\ \citep[\pnum{16}]{levinson_timing_2015}.

Zooming out, turn-taking is not only present in humans.
Figure \ref{fig:species}, based on \citet[\pnum{11}, \fnum{2}]{levinson_turn-taking_2016}, shows a number of related species who are known to employ either vocal or gestural turn-taking.
The original purpose of \citeauthor{levinson_turn-taking_2016}'s figure was to show the presence of turn-taking in non-human primates, it can therefore not be considered exhaustive.
\emph{diaemus youngi}, one of several bat species known to engage in vocal turn-taking \citep[\pnum{114}]{vernes_what_2017}, has been added to show that the phenomenon is in fact further spread in the animal kingdom than the various apes and monkeys might lead one to believe.\footnote{This is not to suggest that \citeauthor{levinson_turn-taking_2016} intentionally or ignorantly omitted non-primate species who engage in turn-taking. The figure in his paper simply serves a different purpose than the one in this thesis.}
Some indications for turn-taking have also been found in European starlings, which have been shown to at least prefer song alternation to song overlap \citep{henry_social_2015}, and there are many more examples.

\citet[\pnum{6}]{levinson_turn-taking_2016} suggests that language followed turn-taking in evolution. He does not consider this a coincidence but furthermore suggests that turn-taking formed the basis for language evolution.
This claim rests on three main indications: a) turn-taking is present in various species related to humans, b) humans commonly employ language together with other modalities present in various non-human primates (e.g.~gesture), and c) language development in infants begins at an early age.

\begin{figure}
	\centering
	\input{../aux/tikzpictures/tikz_species.tex}
	\caption[Species known to engage in vocal or gestural turn-taking.]{A selection of species known to engage in vocal or gestural turn-taking. Based on \citet[\pnum{11}, \fnum{2}]{levinson_turn-taking_2016}.}
	\label{fig:species}
\end{figure}

Taking another step back, this raises the question of how turn-taking itself was selected for in evolution.
From this point of view, turn-taking is more interesting than other language universals, whose evolutionary advantages are quite obvious.
The notion of a subject and the notion of a verb \citep[both considered language universals,][]{robins_noun_1952,hopper_iconicity_1985} allow humans to express a variety of semantic concepts that they may not be able to express without these notions.
The evolutionary advantage of turn-taking, on the other hand, still remains unclear.

Fundamentally, evolution works by means of random mutations in the passing on of an organism's \dna.
The \dna\ of the first child generation entirely depends on the \dna\ of the parent generation\dash with occasional random mutations introduced.
On a non-genetic level, one might rephrase this by stating that, with every new generation, organisms do not make exact copies of themselves, but copies with ever so slight modifications \citep[\cnum{8}]{dediu_introduction_2015}.
This mechanism is what Charles Darwin referred to as ``descent with modification'' \citep[\pnum{116}]{darwin_origin_1958}, a term which is currently experiencing a renaissance in biolinguistics \citep[e.g.][]{boeckx_conjecture_2017}.
These modifications may result in all sorts of differences between a parent and their child, be it less muscle mass, a lighter feather colour, or a more sophisticated sense of hearing.
Naturally, as these modifications are random, they will cancel each other out in the larger picture.
For every bat born with thinner wings, there will be another one with thicker wings; for every human with more red blood cells, there will be another one with fewer red blood cells; and so on.
The only circumstance under which a change can permanently push through in the long run and become the norm in an entire population is if that new trait causes its bearers to produce significantly more offspring than those individuals who do not bear the trait \citep[\pnum{168~ff.}]{dediu_introduction_2015}.
As a result of the significantly higher offspring, the new trait will be present in significantly more individuals in the first child generation, who in turn will pass it on to even more individuals of the second child generation.
Over time, those individuals bearing the trait will outnumber the individuals not bearing the trait until eventually the new trait will be present in all individuals of the species.%
\footnote{This is a rough sketch of a complex process, omitting many details which are less relevant for the issue at hand.
For example, the new trait must have high heredity. This refers to the property of a trait being passed on to a large proportion of the child generation \citep{king_heredity_2013}.
Humans wearing earrings can be fairly certain that their offspring will inherit the property of having two hands.
They can be much less certain that their offspring will also be wearing earrings.
Having two hands is a trait with relatively high heredity; wearing earrings is a trait with relatively low heredity \citep[\pnum{243~ff.}]{sapolsky_behave_2017}.
Furthermore, there are theoretical positions stating that evolution in the sense described here only applies after the percentage of individuals in a population bearing the trait has surpassed a certain threshold.
Below this threshold, the only mechanism at work is chance.
It is also debated whether this threshold is dependant on or independent of the total population size \citep[\pnum{21~f.},\pnum{90--94}]{berwick_why_2016,gillespie_population_2010}.}
This evolutionary advantage is not always straight-forward and easy to spot.
It may be intuitive that a male gorilla with stronger muscles is more successful in a mating battle; much less intuitive that a larger skin surface can take in more warmth from the sun, which is then followed by an easier maintaining of body temperature \emph{(thermoregulation),} leading to better survival in colder climates and consequently more chances of successful mating.
The latter, less intuitive example is a rough sketch of how the first stages of the evolution of wings are hypothesised to have taken place \citep{douglas_thermoregulatory_1981,kingsolver_aerodynamics_1985}.

An implicit necessity of the evolutionary mechanism is that every single evolutionary step along the way must in itself be of use to its bearers.
Sticking with the wing-example, organisms will not develop tiny winglets that become bigger and bigger because somewhere down the line they will become useful for flight.
The initial tiny winglets would never gain a foothold in the population if their bearers were not in advantage.
Every tiny adaptive step in itself must have been useful, be it thermoregulation, aerodynamics, or any other kind of advantage.

How does turn-taking fit in this picture?
Being a characteristic of communication systems\footnote{%
This is not to be misunderstood as taking the stance that language itself is a communication system.
On the contrary, there is substantial evidence that language has not been selected for \emph{as} a communication system \citep{reboul_why_2015}, even though it is obviously suited to be used as such. Turn-taking, however, governs far more than language (e.g.~gestural and vocal turn-taking, cf.~figure~\ref{fig:species}) and all aspects it governs seem to be closely linked to communication and interaction.}%
, it seems an arbitrary, even counterproductive restriction to impose.
If we think of a communication system between two organisms, the a priori assumption would be that information simultaneously flows from \emph{a} to \emph{b} and from \emph{b} to \emph{a}.
This line of thought becomes more intuitive if one imagines humans would make use of turn-taking when laughing.
Not only does this behaviour strike us as odd (as is to be expected, because it is an unnatural behaviour to humans), but it also seems notoriously slow.
And indeed, limiting information flow to only one direction at a time is a considerable disadvantage that must be justified by some other advantage it may be a byproduct of.
Two possible reasons for this restriction come to mind, the first of which unpacks as follows:

If, in a conversation, participant \emph{a} both listens to participant \emph{b} and speaks at the same time, then what will in effect reach \emph{a}'s ear is a mixture of both participants' speech.
\emph{a} is now challenged with the task of disentangling the two sources and isolating \emph{b}'s signal from the received input.
Avoiding this difficult task may well be worth limiting information flow to one direction only.
But for several reasons, this does not seem to be the case.
First off, the capacity to isolate a human voice in an environment of several voices is a capacity that to a certain extent is already present in humans.
In psychoacoustics, it is known as the cocktail party effect \citep[first mentioned in][a more recent review is given in \citeauthor{arons_review_1992}, \citeyear{arons_review_1992}]{pollack_cocktail_1957}.
But even if humans were not capable at all of discerning a single voice in a noisy environment, turn-taking still could not be justified by this line of argument. The described scenario of having to separate two acoustic signals from each other is in fact a peculiarity of the acoustic modality.
It is no challenge at all to \emph{see} another conversation partner's gestures while producing gestures at the same time\dash the visual modality allows such simultaneity.
Consequently, this line of argumentation fails for gestural turn-taking.
Motivating gestural and vocal turn-taking differently would tremendously complicate any theory of turn-taking and should therefore be avoided unless the empirical data contain strong clues hinting in such a direction.
As of now, this is not the case and therefore the sensible assumption is that this line of argumentation will also fail as a cause for vocal turn-taking.%
\footnote{
Explained differently:
It is of course true that for the acoustic modality this line of argumentation is not wrong per se.
But it is in all probability not the \emph{cause} for vocal turn-taking. %%%%%%%%%
Claiming that ``vocal turn-taking is caused by the difficulties of separating two acoustic signals from each other'' is similar to claiming that ``infants decide to stop being breast fed because they cannot survive from nothing but breast milk their entire life''.
It is of course true that breast milk does not contain nutrients suited for an adult human and does not come in quantities sufficient to feed a human their entire life.
But anyone who has ever seen the looks on infants' faces when their parents decide to stop breast feeding knows that this is not a choice the infant approves of.
They stop being breast fed because their parents decide to stop breast feeding them.}
Further support for the claim that this acoustic peculiarity is not responsible for turn-taking is the fact that turn-taking is equally present in sign languages, which make use of the visual modality \citep[see e.g.][among many others]{devos_turn-timing_2015,girard-groeber_management_2015, groeber_turns_2014,manrique_suspending_2015,mcclearly_turn-taking_2013}.

The second possible justification for turn-taking that comes to mind is a competition for resources.
Maybe the human brain is simply not capable of mastering both the production and reception task at the same time.
But again, this cannot be the whole story.
\citet{levinson_timing_2015} analysed the gaps between turns in the Switchboard Corpus, a corpus consisting of American English telephone conversations recorded across the \textsc{usa} \citep{calhoun_nxt-format_2010,godfrey_switchboard_1992}.
They defined a gap, following \citet{heldner_pauses_2010}, as ``portions of the stereo signal that contained silence in each speaker's channel, and that involved a floor transfer between the two speakers'' \citep[\pnum{16}]{levinson_timing_2015}.
This definition is in part unclear because \citet[\pnum{556}]{heldner_pauses_2010} do themselves not define what a \emph{floor transfer} is but merely give the term as one of many in a list of synonyms having been used in the literature\dash all essentially meaning \emph{gap} or \emph{pause}.
However, \citet[\pnum{556}]{heldner_pauses_2010} define a \emph{gap} as ``silences bounded by speech from different speakers'' and it becomes clear from context that this is also the definition \citet{levinson_timing_2015} have in mind.
\citet{levinson_timing_2015} found that the modal gap between turns had a length of only 200~\ms.
In a large meta-analysis of 82 studies on word production, \citet{indefrey_spatial_2004} and \citet{indefrey_spatial_2011} (the latter being a later update in light of recent research advances) state that the production time of a single primed\footnote{%
\emph{Priming}, in linguistics and psychology, refers to drawing an individual's attention to concepts that are similar to a given stimulus and then presenting that stimulus shortly afterwards.
An example of semantic priming is presenting an individual with pictures of fruit and then asking them to name a picture of an apple afterwards.
An example of phonetic priming (in most West- and North-Germanic languages, including English and German) is presenting an individual with a picture of a mouse and then asking them to name a picture of a house afterwards.
Priming has been shown to decrease, but never increase, the time to name the stimulus \citep[\pnum{84~ff.}]{traxler_introduction_2012}.}
word is 600~\ms.
Referring to this meta-analysis, \citet{levinson_turn-taking_2016} points out that this minimal production time is already longer than the modal gap of 200~\ms\ found in \citet{levinson_timing_2015}.
Consequently, speech production must begin before the end of the previous turn.
Or, in other words, this is evidence for the fact that, at least for a finite amount of time, human cognitive resources are sufficient for receiving and planning speech at the same time.
Planning is of course in its cognitive demands not necessarily identical to production\dash the former lacks any articulatory efforts\dash but \citet{levinson_timing_2015} also found negative gaps (overlaps) between turns. This means that humans are also capable of simultaneously comprehending and producing speech (for a finite amount of time).
Despite this temporary ability of simultaneous comprehension and production, competition for resources seems a more viable candidate to justify turn-taking.
Being able to simultaneously produce and receive speech for a timespan that can be reasonably expressed in milliseconds is not at all the same as being able to simultaneously receive and produce speech for the duration of an entire conversation.

Other language universals, such as subjects and verbs, have quite obvious advantages.
They facilitate or even enable us to express a variety of semantic concepts.
For turn-taking, this advantage is far from clear.
While turn-taking is being heavily discussed and debated in the psycholinguistic literature, its investigation in the evolutionary and biolinguistic literature is still in its infancy.
The Oxford Handbook of Language Evolution \citep{tallerman_oxford_2012}, over 600~pages strong and at the time of writing 6 years old, mentions the term a single time.
In a chapter entitled ``Gossip and the social origins of language'', \citet[\pnum{345}, emphasis mine]{dunbar_gossip_2012} states that ``[g]rooming is a strictly one-on-one activity, but, in naturally-forming conversations (as opposed to lecture-like contexts where \emph{turn-taking} is formally regulated by social rules) language allows us to interact with up to three individuals at the same time''.
Other recent works concerned with language evolution that do \emph{not} touch on the issue of turn-taking include \citet{berwick_why_2016}, \citet{bolhuis_birdsong_2013}, \citet{friederici_language_2017}, and \citet{jenkins_biolinguistics_2000}.
We are therefore left with the question of how the turn-taking system was selected for in evolution.
A question which at present lacks sufficient answers.


\section{Previous Research}
\label{sec:introres}
The overview of previous research given here is twofold. Subsection \ref{ssec:introrestt} reviews literature approaching turn-taking as a multimodal phenomenon.
Subsection \ref{ssec:introresfpm} gives a brief overview of the development of \fpmlower.
The computer science literature is mainly concerned with theoretical, algorithmic aspects of \fpmlower, which is not the subject of the present thesis.
Instead, literature on applications of \fpmlower\ in linguistics will be reviewed.


\subsection{Multimodal Turn-Taking}
\label{ssec:introrestt}
The literature considering turn-taking to be a unimodal, linguistic phenomenon is large \citep[to name but a few recent examples]{casillas_turn-taking_2016,freud_turn-taking_2016,heldner_pauses_2010,kurtic_resources_2013}.
Nevertheless, one can find examples of multimodal turn-taking, although those studies are much fewer in number.
\citet{levinson_turn-taking_2016} indirectly hints at multimodality.
He establishes turn-taking as a universal feature of language that ``[a]ppear[s] earlier in ontogeny than linguistic competence'' \citep[\pnum{6}]{levinson_turn-taking_2016}.
He further states that ``turn-taking thus involves multi-tasking comprehension and production, but \emph{multi-tasking in the same modality is notoriously difficult}'' \citep[\pnum{9}, emphasis mine]{levinson_turn-taking_2016}.
Being concerned with cognitive load, this may be interpreted as an argument in favour of multimodal turn-taking, although it is admittedly not clear that this is what \citeauthor{levinson_turn-taking_2016} is hinting at.
Turn-taking is not explicitly described as multimodal in his paper.

The literature on multimodal turn-taking can roughly be divided into two groups.
The first group are studies examining turn-taking in infants before the acquisition of the first word.
Though many of these studies are not concerned with multimodal turn-taking (and some do not even mention the fact that pre-language vocalisation is not at all equal to using language), they do provide evidence for the fact that infants engage in turn-taking before being able to make use of the linguistic modality.
The second group are studies whose participants already have acquired language to an extent that they can actively use it for entire interactions.
These studies do take non-linguistic modalities into account, the most common ones being gesture, gaze, and smile. %justify this order, mention used modalities for every cited study

\paragraph{Turn-taking before the acquisition of the first word}
% gratier 2015
\citepos{gratier_early_2015} study suggests that infants as young as 2 months actively engage in turn-taking.
This is long before first words are acquired, which usually happens around one year of age \citep[\pnum{129}, \pnum{131}]{lenneberg_biological_1967,szagun_sprachentwicklung_2013}.\showcomment{Fucking citation bug!}
They compared interactions between mothers and infants of 8--13~weeks and 17--21 weeks of age respectively.
Only the acoustic modality was taken into account.
Infants' turns were considered to be \emph{latched} when the time between the end of the mother's turn and the onset of the infant's turn was less than 50~\ms.
The percentage of latched turns between mother and infant was 44.5~\%, with insignificant variation between the two age groups.
This suggests not only active participation in turn-taking by the infant (44.5~\% being far above chance), but also a \emph{constant} (i.e.\ not improving with age) turn-taking ability, at least between the second and fifth month of life.
Yet, they also state that ``there remains some controversy over the extent to which young infants actively contribute to turn-taking exchanges and the extent to which adults construct conversational frameworks for infant vocalization'' \citep[\pnum{2}]{gratier_early_2015}, but without going further into what these controversies are.
Even though this study does not consider any non-acoustic part of the interaction, it still provides evidence for turn-taking at an age of only 2 months.

% hilbrink 2015
\citepos{hilbrink_early_2015} longitudinal study investigates turn-taking in infants of 3 to 18 months of age.
In contrast to \citet{gratier_early_2015}, they defined a turn-transition\footnote{Essentially a synonym for what is here called turn-taking.} as ``any switch from a maternal utterance to an infant vocalization or vice versa'' \citep[\pnum{5}]{hilbrink_early_2015} without any temporal conditions in their definition and found a median gap duration ranging from 500 to 1200~\ms, depending on infant age.
Interestingly, the highest median gap duration was shown by infants at 9~months of age.
\citet{hilbrink_early_2015} give two possible reasons for this peculiarity:
The first possibility ascribes the increasing duration to the mothers' behaviour.
With increasing age of their infant, mothers expect their infant more and more to actively engage in turn-taking.
Therefore, when mothers finish a turn, the time they wait before starting to speak again increases.
This increasing pause is intended as an opportunity for the infant to take over and ``take the turn''.
This behaviour will result in longer median gap durations even if the infant does start vocalising at entirely random times.
If a mother waits 300~\ms\ after finishing her turn before starting to speak again, then the maximum possible gap the infant can reach by producing turns at random times is exactly 300~\ms.
If, on the other hand, the mother waits for 1200~\ms\ before starting to speak again, then the infants' (alleged) random turns can produce gaps as long as 1200~\ms.
If this line of argumentation holds true, then the infant is not at all involved in the increasing median gap durations.
The second possible explanation is that with increasing linguistic ability, the amount of parsing infants have to do before beginning their turn increases and\dash consequently\dash the mean gap duration increases.
\citet[\pnum{10}]{hilbrink_early_2015} make no definitive statement as to which of the two possibilities they hypothesise to be true, but they heavily lean towards the second possibility.

Both \citet{gratier_early_2015} and \citet{hilbrink_early_2015} only take infants' vocalisations and mothers' speech into account\dash other modalities do not play a role in their analysis.
They also employ very different temporal definitions for turn-taking.
While \citet{gratier_early_2015} take the high percentage of turns with a gap of less than 50~\ms\ as evidence for turn-taking, \citet{hilbrink_early_2015} measure gaps of more than one second and consider it an instance of turn-taking.
I do not intend to suggest that such long gaps are not cases of turn-taking, but the difference in operationalisation indicates a lack of consensus in the literature about the temporal characteristics turn-taking must fulfil in order to be considered turn-taking.


\paragraph{Turn-taking including non-linguistic modalities}
% important multimodal studies
% Gogate et al (2000)
\citet{gogate_study_2000} make an important contribution to multimodal research.
In 24 dyads during semi-structured play, mothers taught there infants four novel words.
Speech, gesture, and touch in the mothers' utterances were taken into account.
With this coding scheme, 99.99\% of the mothers' utterances were multimodal.
\citet{gogate_study_2000} found that mothers used novel words significantly more often in synchrony with gesture/touch than known words.
More specifically, mothers tended to move the referent object while naming it.
This seemingly redundant information is hypothesised to help infants make the arbitrary connection between a word's acoustic form and its semantic function \citep[\pnum{891}]{gogate_study_2000}.
This makes multimodality a fundamental aspect of mother-infant interaction.

% Van Egeren et al (2001)
\citet{vanegeren_mother-infant_2001} inspected response timing in early mother-infant interaction using a sample of 150 mother-infant dyads (with infants being 4~months of age) engaging in naturalistic mother-infant play.
They took into account speech and vocalisation, but also smile, touch, social play, and object play.
The object of interest was the delay between a signal and a response, where a signal is any of the coded behaviours of mother or infant and a response is a behaviour of the other party beginning with 1--5~\s\ after the signal had begun.
Within one \sone\ after one party had exhibited a signal, the likelihood of the other party exhibiting a response was not above chance.
But, interestingly, within 2--5~\s\ after one party had exhibited a signal, the likelihood of the other party's response ``increased considerably and significantly for virtually all signal-response pairs'' \citep[\pnum{687}]{vanegeren_mother-infant_2001}.

% less important multimodal studies
A considerable number of studies has taken gesture into account when investigating turn-taking \citep[among others]{benus_pragmatic_2011,mccowan_automatic_2005,schmitt_zur_2005,stivers_universals_2009}, but not all of these are specifically concerned with multimodality and not all of them consider gesture and speech separately in their analyses (e.g.~in \citet{stivers_universals_2009} both gesture and speech are considered as valid beginnings of a turn, but the two modalities are not analysed separately).
\citet{mondada_participants_2006,mondada_multimodal_2007} investigates multimodal models for assessing how speakers choose the next speaker in an interaction of more than two speakers.

% tt in sign languages is not multimodality
One might be inclined to consider the presence of turn-taking in sign languages (see page \showcomment{PAGE}) as a further argument for multimodal turn-taking.
I argue against such a claim.
While sign languages are obviously not acoustic and could be described as motor-visual interaction, it still is a \emph{linguistic} interaction.
Sign languages have a syntax identical to that of acoustic human languages and show all other key characteristics that are known in acoustic languages \citep{stokoe_sign_2005}.

% turn-taking in cs
Within computer science, \citet{chan_designing_2008} criticise the unimodality of then-existing computational models of turn-taking, stating that ``existing turn-taking protocols are rudimentary, lacking many communication cues available in face-to-face collaboration'' \citep[\pnum{333}]{chan_designing_2008}.
In the following years, there has been some research in multimodal turn-taking within (broadly) robotics and artificial intelligence.
\citet{chao_simon_2011}, \citet{chao_timing_2012}, and \citet{chao_timingmultimodal_2012} create computational turn-taking models (employing speech, gaze, and motion) for humanoid robots to improve communication with humans.
\citet{huang_multimodal_2011} create a model for giving virtual humans the ability to predict turn ends.
\citet{bohus_multiparty_2011}\dash building on \citet{bohus_computational_2010,bohus_facilitating_2010}\dash improve a multiparty model for turn-taking, where several people can interact with a digital avatar as participants in a quiz game hosted by the digital avatar.
In its current state, the model is capable of ``visually tracking multiple people in the scene, sound source localization, speech recognition, conversational scene analysis, behavioral control and dialog management'' \citep[\pnum{100}]{bohus_multiparty_2011}.




\subsubsection{Gaze in Interaction}
\subsubsection{Smile in Interaction}
\showcomment{Priority low, only if there's time.}

\subsection{(Frequent) Pattern Mining}
\label{ssec:introresfpm}
There have been some applications of pattern mining within linguistics.
But before reviewing this literature, a few words on terminology are in order.
Pattern mining was first proposed by \citet{agrawal_mining_1993}, originally intended for marketing purposes answering questions of the sort:
If a customer purchases item \emph{a}, which item are they likely to purchase next?
How likely are they to also purchase a given item \emph{b} \citep[\pnum{56}]{han_frequent_2007}?
\emph{Frequent} pattern mining is a certain kind of pattern mining, that looks for patterns in a data set that appear \emph{frequent}.
Here, \emph{frequent} essentially means at least as often as a user-defined (i.e.~essentially arbitrary) threshold.
If, for example, the user-defined threshold \emph{s} is defined as \emph{s}~=~3, then patterns occurring twice or less in the data set are not taken into account \citep[\pnum{280}]{han_data_2012}.
\emph{Sequential} pattern mining, first mentioned in \citet{agrawal_mining_1995}, is a kind of pattern mining that looks for sequences, i.e.~ordered item sets.
The question here is no longer how often \emph{a, b,} and \emph{c} are contained in a transaction in the data set, but instead how often \emph{a} followed by \emph{b} (followed by \emph{c}, etc) is contained in the data set.
In more formal terms, sequential pattern mining is looking for ordered sets\dash called sequences\dash such as <\emph{a,~b,~c}>, where \emph{a,~b,~c} may essentially be any data structure \citep[\pnum{589}]{han_frequent_2007}.
This very brief introduction already shows that the different kinds of pattern mining are not mutually exclusive.
One could for example mine for sequences of a given kind that appear at least as often as a given threshold and would thereby be applying both \fpmlower\ and sequential pattern mining at the same time.
For both this reason as well as the fact that the literature itself is quite limited, the literature reviewed in this subsection is not restricted to applications of \emph{frequent} pattern mining, but pattern mining in general.

Within linguistics, the only existing application of pattern mining is the so-called text mining, which could hardly be further away from the present application.
Text mining refers to ``the process of extracting interesting information and knowledge from unstructured text'' \citep[\pnum{19}]{hotho_brief_2005}. Within text mining, some of the studies making use of more inherently linguistic theories (e.g.~using a syntax model a linguist would consider adequate) are \citet{bechet_discovering_2012} and \citet{quiniou_what_2012}.
The former use a sequence mining approach on two text corpora, one consisting of articles in the portrait section of the French newspaper \emph{Le Monde}, the other consisting of articles in the arts section of the same newspaper.
Their aim is to mine \emph{appositive qualifying phrases}, which they define as ``phrases denoting judgment or sentiment \citeellipses\ and more generally qualification'' \citep[\pnum{155}]{bechet_discovering_2012} and later as ``phrases expressing judgement or qualification'' \citep[\pnum{159}]{bechet_discovering_2012}.
Their application therefore lies within text linguistics and pragmatics, two subdisciplines with little connection to the present thesis.
\citet{quiniou_what_2012} pursue a similar albeit more general goal.
The authors explore the use of mining techniques for stylistic analysis of texts by applying such tools to three corpora of poetry, letters, and fiction respectively \citep[\pnum{170}]{quiniou_what_2012}.\footnote{%
The similarity of their approach compared to \citet{bechet_discovering_2012} is not surprising, given that all authors except one come from the \textsc{greyc} research lab at the Université de Caen Basse-Normandie and the \textsc{irisa-insa} research lab at the Campus de Beaulieu and that Cellier and Charnois are co-authors of both.}
Interestingly, in both studies the last part of the methodology applied involves having a linguist qualitatively evaluate the quantitative data.
This is similar to the approach applied in the present thesis.


\section{The Aim of this Thesis}
\label{sec:introaim}
The literature reviewed in the previous section has already given many arguments for the fact that turn-taking is a multimodal phenomenon (it is acquired at an earlier age than language, difficulties of unimodal multi-tasking, etc).\showcomment{Find more studies, give more arguments!}
Examining a multimodal phenomenon is much more complex than examining a unimodal one.
A hypothesis-generating technique such as \fpmlower\ is a useful step in revealing sensible research gaps and prospects for potential studies.
The mining scheme applied here and in \citet{rohlfing_multimodal_underreview} is especially well-suited to finding temporal relationships between different modalities.
Once the algorithm has run though, the task of deciding which rules deserve attention and which do not suddenly becomes a very qualitative one.
It would be desirable to have quantitative, numerical support for the qualitative task of choosing which rules to dedicate more attention to.

This is where the present thesis comes in.
By introducing a notion of statistical significance, quantitative numerical support is added to the qualitative task of choosing rules.
The goal is to have 6~values for every observed rule:
3 metrics (confidence, duration, and number of occurrences; all to be explained in section~\ref{sec:fpmmetr}) as well as 3 corresponding \pv s, which will put the three metrics in context.
Qualitative hunches can then be confirmed (or disputed) by objective numbers.
Comparing the actual observed sequences with randomised sequences, the \pv s will be interpretable in a way somewhat similar to responding to questions such as:
Is the relationship between these two modalities an actual coordination, or just the result of chance?






















