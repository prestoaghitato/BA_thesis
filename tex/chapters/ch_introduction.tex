% !TEX root = ../ba_master.tex
% @author Marcel Ruland (2018)

\chapter{\introduction}
\label{ch:introduction}
A \emph{turn} is a basic unit of speech in conversation. When a speaker begins their turn, they gain the right to speak and\dash upon finishing their utterance\dash pass that right to another participant of the conversation. A turn is on average 2~sec.\ in length with very high variation. They usually consist of syntactically complete clauses and give pragmatically sufficient information, though this is by no means a necessity \citep[\pnum{8}]{levinson16}.
\emph{Turn-taking} is the act of one speaker finishing their turn and another speaker beginning their turn, i.e.\ passing on the right to speak from one speaker to another. Turns must not necessarily be strictly linguistic\footnote{By \emph{strictly linguistic} I here essentially mean ``making use of language''.}, but may make use of, for example, gestures \citep{missingsource}. In fact, children regularly engage in turn-taking before having acquired their first words \citep[\pnum{1311}]{casillas16}, i.e.\ turn-taking is not intrinsically linguistic in and of itself.

If turn-taking is not dependant on linguistic means but may make use of such, then it logically follows that turn-taking must be a multimodal phenomenon. Despite this fact, turn-taking has mostly been considered as unimodal in the literature \citep[\pnum{1}]{rohlfing18}. The downside of investigating only one modality of a multimodal phenomenon is\dash quite obviously\dash that it ``omits other behaviors relevant to the exchange'' \citep[\pnum{3}]{rohlfing18}.

\citet{rohlfing18} have identified and approached turn-taking as a multimodal phenomenon; taking into consideration linguistic utterances, non-linguistic vocalisations, gaze, and smile. They did so by applying \fpmlower\dash an explorative methodology at the intersection of computer science and statistics\dash to a corpus of 11 mother-infant dyads.

The present thesis aims to strengthen the statistical objectiveness of applying \fpmlower\ to human interaction. More specifically, it does so by introducing a notion of significance. The thesis is organised in four main chapters.
Chapter \ref{ch:introduction} (i.e.\ the current chapter) will proceed by giving an overview of past research and then further clarifying the aim of this thesis.
The second chapter ``\frequentpatternmining'' explains the entire process of preparing the data, applying \fpmlower, and then evaluating the data\dash beginning with a description of the raw video material and ending with abstract rules\footnote{The notion of a \emph{rule} will be explained in subsection \ref{sec:mining}.}.
Chapter \ref{ch:significance} ``\significance'' introduces a method for establishing statistical significance in the data. Both chapters \ref{ch:mining} and \ref{ch:significance} present results in their last sections.
Chapter \ref{ch:visualisation} ``\visualisation'' is of a different nature. In contrast to the two previous chapters, it is not concerned with methods of evaluating the data but with methods of making the found results easily comprehensible for humans by means of visualising them in an adequate way.

\section{Previous Research}
The overview of previous research given here is twofold, coming both from a turn-taking and from a \fpmlower\ perspective. \citet{rohlfing18} is as of now the only study combining the two. I have therefore not included a section of previous research on this specific application.


\subsection{Turn-Taking}
\citep{levinson16} transition between speakers is three times faster than language encoding, psycholinguistics has studied language production and comprehension separately
%turn taking has so far been considered unimodal, until \citet{rohlfing18} came along

\subsection{\fpmupper}
%brief explanation \fpmlower, there is only one single study so far


\section{The Aim of this Thesis}
The present study builds on work by \citet{rohlfing18}, who applied \fpmlower\ techniques to a corpus of mother-child dyads, thereby identifying turn-taking as a multimodal phenomenon. It aims to further improve and operationalise this methodology for behavioural studies of similar sort.
introduce notion of significance, visualise significant patterns


























%\section{Pattern Mining Compared to Traditional Statistics}
%Statistical methods can broadly be divided into two subgroups. \emph{Inferential statistics} is concerned with methods suited to affirm or reject a given hypothesis in a data set, where the hypothesis is formulated in advance. A canonical example are significance tests, where we formulate a null-hypothesis and then perform tests to determine whether it can be affirmed\footnote{``affirmed'' in the sense of ``not proven, but no evidence against it either''} or rejected. \emph{Exploratory statistics} on the other hand do create new hypotheses. With no hypothesis formulated in advance, the tests are performed on a data set and the results may be used to formulate new hypotheses afterwards. A canonical example hereof is the relatively young field of \fpmlower.