% @author Marcel Ruland (2018)

\chapter{Introduction}
A \emph{turn} is a basic unit of speech in conversation. When a speaker begins their turn, they gain the right to speak and\dash upon finishing their utterance\dash pass that right to another participant of the conversation. \emph{Turn-taking} is the act of one speaker finishing their turn and another speaker beginning their turn, i.e.\ passing on the right to speak from one speaker to another \cite[\pnum{8}]{levinson16}.

\section{Previous Research}
\cite{levinson16} transition between speakers is three times faster than language encoding, psycholinguistics has studied language production and comprehension separately
%turn taking has so far been considered unimodal, until \citeasnoun{rohlfing18} came along
%brief explanation \fpm, there is only one single study so far

\subsection{Turn-Taking}
\cite{stivers09}

\subsection{\fpm}



\section{The Aim of this Study}
The present study builds on work by \citeasnoun{rohlfing18}, who applied \fpm{} techniques to a corpus of mother-child dyads, thereby identifying turn-taking as a multimodal phenomenon. It aims to further improve and operationalise this methodology for behavioural studies of similar sort.
introduce notion of significance, visualise significant patterns


























%\section{Pattern Mining Compared to Traditional Statistics}
%Statistical methods can broadly be divided into two subgroups. \emph{Inferential statistics} is concerned with methods suited to affirm or reject a given hypothesis in a data set, where the hypothesis is formulated in advance. A canonical example are significance tests, where we formulate a null-hypothesis and then perform tests to determine whether it can be affirmed\footnote{``affirmed'' in the sense of ``not proven, but no evidence against it either''} or rejected. \emph{Exploratory statistics} on the other hand do create new hypotheses. With no hypothesis formulated in advance, the tests are performed on a data set and the results may be used to formulate new hypotheses afterwards. A canonical example hereof is the relatively young field of \fpm.